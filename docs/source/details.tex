---
header-includes: |
    \usepackage{amsmath}
    \usepackage{physics}
---

# Details of computation

We present the equations necessary to reproduce the code. For a more complete
description of the theory, see[@PhysRevC.89.065501].

## Differential event rate

The main computation of the code is the differential event rate for WIMP-nucleus
scattering events. This is obtained by integrating the differential WIMP-nucleus
cross section over the velocity distribution of the WIMP-halo in the galactic
frame: \begin{equation}\label{ER}
\begin{split}
	\frac{dR}{dE_r}(E_r)
	 = N_T n_\chi \int_{v_{min}}^{\infty} \frac{d\sigma}{dE_r}(v,E_r)\ \tilde{f}(\vec{v})\ v\ d^3v,
\end{split}
\end{equation}
where $E_r$ is the recoil energy of the WIMP-nucleus scattering event, $N_T$ is
the number of target nuclei, $n_\chi = \rho_\chi/m_\chi$ is the local dark
matter number density, $\sigma$ is the WIMP-nucleus cross section.  The dark
matter velocity distribution in the lab frame, $\tilde{f}(\vec{v})$, is obtained
by boosting the Galactic-frame distribution $f(\vec{v})$:
\begin{equation}
    \tilde{f}(\vec{v}) = f(\vec{v} + \vec{v}_E),
\end{equation} 
where $\vec{v}_{E}$ is the velocity of the earth in the galactic rest frame.

## Standard Halo Model

There are many models for the dark matter distributions of galaxies
[@RevModPhys.85.1561].  We provide the simplest model, a three-dimensional
Maxwell-Boltzmann distribution,
\begin{equation}
	f(\vec{v}) \propto e^{-\vec{v}^2/v_0^2},
\end{equation}
where $v_0$ is some scaling factor (typically taken to be around $220$ km/s).
This is called the Simple Halo Model (SHM) when a maximum value of the speed,
due to the galactic escape velocity $v_{escape}$, is taken into account
[@PhysRevD.33.3495; @PhysRevD.37.3388]:
\begin{equation}
    f_{SHM}(\vec{v}) = \frac{\Theta(v_{esc}-|\vec{v}|)}{\pi^{3/2}v_0^3N_{esc}} 
    e^{-(\vec{v}/v_0)^2},
\end{equation}
where $N_{esc}$ renormalizes due to the cutoff:
\begin{equation}
    N_{esc} = erf{(v_{esc}/v_0)} - \frac{2v_{esc}}{\sqrt{\pi}v_0}
    \exp{-(v_{esc}/v_0)^2}.
\end{equation}
With this distribution, the integral in the differential
event-rate has the form: 
\begin{equation}
    I_{MB} = \int_{\Omega} d^3v \frac{d\sigma(v,q)}{d{q}^2}\ v\ e^{-(\vec{v}+\vec{v}_E)^2/v_0^2},
\end{equation}
where the constraint $\Omega$ is that $v_{min}^2<(\vec{v}+\vec{v}_E)^2<v_{esc}^2$.

To reduce to a one-dimensional integral, we make the conversion to spherical
coordinates. Special care has to be taken to properly handle the truncated domain. 
Noting that $(\vec{v}+\vec{v}_{earth})^2 = \vec{v}^2 +
\vec{v}^2_{earth} + 2vv_{earth}\cos(\theta)$, with $\theta$ defining the angle
between the two vectors, we make the substitution $d^3v = d\phi d(\cos \theta)
v^2 dv$. 

It can be shown that[^integral]:
\begin{equation}
    \int_{(\vec{v}+\vec{v}_E)^2<v_{esc}^2} d^3v = \int_0^{2\pi}d\phi \left[
    \int_0^{v_{esc}-v_E}v^2dv\int_{-1}^{+1}d\cos{\theta} 
    + \int_{v_{esc}-v_E}^{v_{esc}+v_E}v^2dv\int_{-1}^{(v_{esc}^2-v_E^2-v^2)/2vv_E}d\cos{\theta}\right].
\end{equation}
Making the physically justified assumption that $v_{esc}-v_E > v_{min}$, we can
simply shift the limit on the first integral from $0$ to $v_{min}$. Along the
way we will need to work out the angular integrals:
\begin{equation}
\int_{-1}^{+1}d\cos{\theta} e^{-2vv_E\cos\theta/v_0^2}=
-\frac{v_0^2}{2vv_E}\left(e^{-2vv_E/v_0^2}-e^{2vv_E/v_0^2}\right);
\end{equation}
\begin{equation}
\int_{-1}^{(v_{esc}^2-v_E^2-v^2)/2vv_E}d\cos\theta e^{-2vv_E\cos\theta/v_0^2}=
-\frac{v_0^2}{2vv_E}\left(e^{-(v_{esc}^2-v^2-v_E^2)/v_0^2}-e^{2vv_e/v_0^2}\right).
\end{equation}
Combining all of this together, and simplifying, we obtain a one-dimensional
integral which we can evaluate with quadrature:
\begin{equation}
I_{MB} = \int_{v_{min}}^{v_{esc}+v_E} dv\ \frac{d\sigma(v,q)}{d{q}^2} v^2 \frac{\pi v_0^2}{v_e} \left\{ \Theta_{v<v_{esc}-v_e}\left[g(v-v_E)-g(v+v_E)\right] + \Theta_{v>v_{esc}-v_E}\left[g(v-v_E)-g(v_{esc})\right]\right\},
\end{equation}
where 
\begin{equation}
g(v) = \exp(-v^2/v_0^2).
\end{equation}


[^integral]: This can be deduced from a geometrical argument: Imagine
    constructing $\vec{v}_E+\vec{v}$. There are three cases to consider depending
    on the size of $\vec{v}$, and the implications for the allowed angles $\theta$
    between $\vec{v}_E$ and $\vec{v}$ that satisfy the constraint 
    $(\vec{v}_E+\vec{v})^2<v_{esc}^2$. Case 1: "Small v", which we define as
    $v<v_{esc}-v_E$. Here there are no restrictions on $\theta$ since by
    construction the magnitudes together cannot exceed $v_{esc}$, so $\cos\theta$
    is limited only by $-1$ and $+1$. Case 2: "Medium v", in which now 
    $v>v_{esc}-v_E$, so not all angles are allowed. To keep the sum from
    exceeding $v_{esc}$, the angle must be restricted such that 
    $\cos\theta<(v_{esc}^2-v_E^2-v^2)/2vv_E$. Case 2 also requires that
    $v<v_{esc}+v_E$ because we reach... Case 3: "Big v": It becomes 
    impossible to satisfy the restriction once $v>v_{esc}+v_E$.

We then use Gauss-Legendre quadrature to evaluate $I$[^quad]. While there are
analytic solutions for specific velocity-dependences of the cross
section[@LEWIN199687; @PhysRevD.82.075004; @PhysRevD.82.023530], our
implementation favors a model-independent framework without the need to lock-in
a particular form for the WIMP-nucleus cross section. 

The limits of the integral, $v_{min}$ and $v_{esc}$, have physical constraints.
The minimum speed is defined by the minimum recoil energy of a WIMP-nucleus
collision at a momentum transfer $q$: 
\begin{equation} 
v_{min} = q/(2\mu_T),
\end{equation} where
$\mu_T=m_Tm_\chi/(m_T+m_\chi)$ is the reduced mass of the WIMP-nucleus system.
To use the simple Maxwell-Bolztmann distribution approximation, the maximum
speed is taken to be $\infty \approx 12 \times v_{0}$. Otherwise, the maximum
speed is taken to be the galactic escape velocity: $v_{esc} \approx 550$ km/s.

Note that as a function of momentum $q$, the integral is guarenteed to go to
zero above some maximum momentum $q_{max}$. This happens when $v_{min} =
v_{max}+v_{earth}$, which corresponds to:
\begin{equation}
q_{max} = 2\mu_T (v_{max}+v_{earth}),
\end{equation}
\begin{equation}
E_{R,max} = q_{max}^2/2m_T = 2\mu_T^2v_{max}^2/m_T.
\end{equation}
With 150 GeV WIMPs and $^{29}$Si, for example, $\mu_T=23.031916$ GeV,
$m_T=27.209888$ GeV, $v_{max}=550$ km/s $=0.001834602$ GeV/c:
$E_{R,max} = 265.2987$ keV.

[^quad]: While there are analytic solutions for this integral in the form of 
    error functions; we use quadrature since it makes easy to later modify the 
    velocity distribution. For example, adding a velocity cut-off is as easy as
    changing the limit on the quadrature, with no need to write a whole new 
    subroutine.

## Smooth SHM
This distribution appears in the Mathematica code DMFormFactor under the name 
"MBcutoff". It is simply a smoothed version of the SHM. The distribution is:
\begin{equation}
    f_{sSHM}(\vec{v}) = \frac{\Theta(v_{esc}-|\vec{v}|)}
        {N_{esc}\pi^{3/2}v_0^3}
        \left\{\exp[-(\vec{v}/v_0)^2] - \exp[-(v_{esc}/v_0)^2] \right\}.
\end{equation}
Thbeginormalization factor is:
\begin{equation}
    N_{sesc} = erf(z) - \frac{2}{\sqrt{\pi}}z(1+\frac{2}{3}z^2)e^{-z^2},
\end{equation}
$z=v_{esc}/v_0$.
This essentially just adds an additional term to the integral $I_{MB}$, which we
will call $I_{S}$, so that $I=(I_{MB}-I_{S})/N_{sesc}$. Following the same steps
as for $I_{MB}$, we find that 
\begin{equation}
I_{S} = \int_{v_{min}}^{v_{esc}+v_E} dv\ \frac{d\sigma(v,q)}{d{q}^2} v^2 2\pi g(v_{esc})\left[\Theta_{v<v_{esc}-v_e} 2v + \Theta_{v>v_{esc}-v_e} \frac{1}{2v_E}(v_{esc}^2-(v-v_E)^2)\right]
\end{equation}

## More Sophisticated Halo Models
The SHM is actually not a very good model. For example, it ignores the annual
modulation of the Earth's speed through the galanctic frame due to its orbit
around the Sun. Another point is that the cut-off model where the speed
probability drops to zero after an escape velocity is reached is not very
realistic: actually, the cutoff should be smoother. 

A semi-recent review on the subject can be found here[@RevModPhys.85.1561].

## Differential cross section

The differential scattering cross section is directly related to the scattering
transition probabilities $T(v,q(E_r))$: 
\begin{equation}
    \frac{d\sigma}{dE_r}(v,E_r) = 2m_t\frac{d\sigma}{dq^2}(v,q) = 2m_T\frac{1}{4\pi v^2} T(v,q).
\end{equation} 
The momentum transfer $q$ is directly related to the recoil energy by
$q^2=2m_tE_r$, where $m_t$ is the mass of the target nucleus in GeV$/c^2$.


## Transition probability

The WIMP-nucleus scattering event probabilities are computed as a sum of squared
nuclear-matrix-elements:
\begin{equation}
    T(v,q) = \frac{1}{2j_\chi +1}\frac{1}{2j_T+1} \sum_{M_i M_f} \left |\bra{j_TM_f} \mathcal{H} \ket{j_TM_i}\right |^2.
\end{equation}
Here $v$ is the speed of the WIMP in the lab frame, $q$ is the momentum
transferred in the collision, and $j_\chi$ and $j_T$ are the intrinsic spins of
the WIMP and target nucleus, respectively, while  $\mathcal{H}$ is the
WIMP-nucleus interaction,
\begin{equation}
    \mathcal{H} = \sum_i\sum_{x=p,n}c_i^x\mathcal{O}_i^x.
\end{equation}
given in terms of 15 non-relativistic operators $\mathcal{O}_i$. The sum over
$x$ indicates the separate coupling to protons and neutrons. These operators,
listed in \ref{sec: EFT}, are those  constructed to leading order from
\begin{equation}
    i\frac{\vec{q}}{m_N},\ \vec{v}^\perp,\ \vec{S}_\chi,\ \vec{S}_N.
\end{equation}

The transition probability is ultimately factorized to group the operators into
forms familiar from electro-weak theory in a shell model framework
[@DONNELLY1979103]. Doing so yields a folding of two factors: one containing
the EFT content, labeled $R_i^{x,x'}$, and another containing the nuclear
response functions, labeled $W_i^{x,x'}$ for each of the $i=1,...,8$ allowed
combinations of electro-weak-theory operators, discussed in the next section.
One obtains the following form[@Fitzpatrick_2013;@Anand_2014]:
\begin{equation}
    T(v,q) = \frac{4\pi}{(4m_T)^2} \frac{1}{2j_T+1} \sum_{x=p,n}\sum_{x'=p,n}\sum_{i=1}^8 R_i^{x,x'}(v^2,q^2)W_i^{x,x'}(q).
\end{equation}

## Dark matter response functions
\begin{equation}
   R_{M}^{xx'}(v,q) = \frac{1}{4}cl(j_\chi) [ v^{\perp 2} (c_5^{x}c_5^{x'}q^2 + c_8^{x}c_8^{x'}) + c_{11}^{x}c_{11}^{x'}q^2 ]\\
    + (c_1^{x} + c_2^{x}v^{\perp 2} ) (c_1^{x'} + c_2^{x'}v^{\perp 2} )  
\end{equation}
 
\begin{equation}             
R_{\Sigma''}^{xx'}(v,q) = \frac{1}{16}cl(j_\chi) [c_6^{x}c_6^{x'}q^4 + (c_{13}^{x}c_{13}^{x'}q^2 + c_{12}^{x} c_{12}^{x'} ) v^{\perp 2}\\
    + 2c_4^xc_6^{x'}q^2 + c_4^xc_4^{x'}] + \frac{1}{4}c_{10}^xc_{10}^{x '}q^2
\end{equation}    
\begin{equation}    
R_{\Sigma'}^{xx'}(v,q) = \frac{1}{32} cl(j_\chi) [ 2c_{9}^{x}c_{9}^{x'}q^2 + ( c_{15}^{x}c_{15}^{x'}q^4 + c_{14}^{x}c_{14}^{x'}q^2 \\
    - 2c_{12}^{x}c_{15}^{x'} q^2 + c_{12}^{x}c_{12}^{x'}) v^{\perp 2} + 2c_{4}^{x}c_{4}^{x'} ]\\
    +\frac{1}{8}(c_{3}^{x}c_3^{x'}q^2 + c_{7}^{x}c_{7}^{x'})v^{\perp 2}
\end{equation}        
\begin{equation}        
R_{\Phi''}^{xx'}(v,q) = \frac{q^2}{16m_N^2}cl(j_\chi) (c_{12}^x - c_{15}^{x}q^2)(c_{12}^{x '}-c_{15}^{x '}q^2 ) \\
    + \frac{q^4}{4m_N^2}c_3^x c_3^{x'} 
\end{equation}    
\begin{equation}    
R_{\tilde{\Phi}'}^{xx'}(v,q) = \frac{q^2}{16m_N^2}cl(j_\chi)(c_{13}^xc_{13}^{x'}q^2 + c_{12}^x c_{12}^{x'})
\end{equation}        
\begin{equation}        
R_{\Delta}^{xx'}(v,q) = \frac{q^2}{4m_N^2}cl(j_\chi) (c_{5}^{x}c_{5}^{x'}q^2 + c_{8}^{x}c_{8}^{x'}) \\
    + 2\frac{q^2}{m_N^2}c_{2}^{x}c_{2}^{x'}v^{\perp 2}
\end{equation}        
\begin{equation}        
R_{\Sigma' \Delta}^{xx'}(v,q) = \frac{q^2}{4m_N}cl(j_\chi) (c_{4}^{x}c_{5}^{x'} - c_{8}^{x}c_{9}^{x'}) - \frac{q^2}{m_N} c_{2}^{x}c_{3}^{x'} v^{\perp 2}
\end{equation}        
\begin{equation}        
R_{M\Phi''}^{xx'}(v,q) = \frac{q^2}{4m_N}cl(j_\chi)c_{11}^{x} (c_{12}^{x'} - c_{15}^{x'} q^2) \\
    + \frac{q^2}{m_N}c_{3}^{x'}  (c_{1}^{x} + c_{2}^{x} v^{\perp 2})
\end{equation}        
It should be noted that the last two dark matter responses are composed entirely of interference terms, which is to say, they do not come into play unless certain combinations  of EFT coefficients are simultaneously active. These are the coefficient pairs listed in Section \ref{validate}. For example, $c_4$ and $c_5$ together will activate $R_{\Sigma' \Delta}$, but not alone.

## Operators
The WIMP-nucleus interaction is defined by the user in terms of an effective
field theory Lagrangian, specified implicitly by fifteen operator coupling
constants $c_i^x$, (for $i=1,...,15$), where $x=p, n$ for coupling to protons or
neutrons individually.

The code uses the EFT coefficients in explicit proton-neutron couplings, i.e.
the interaction is defined by:
\begin{equation}
    \mathcal{H} = \sum_{x=p,n}\sum_{i=1,15} c^x_i \mathcal{O} _i
\end{equation}
and the 15 momentum-dependent operators are:

\begin{align}
    \mathcal{O} _1 &= 1_\chi 1_N\\
    \mathcal{O} _2 &= (v^\perp)^2\\
    \mathcal{O} _3 &= i\vec{S}_N \cdot \left(\frac{\vec{q}}{m_N}\times
        \vec{v}^\perp\right)\\
    \mathcal{O} _4 &= \vec{S}_\chi \cdot \vec{S}_N\\
    \mathcal{O} _5 &= i\vec{S}_\chi \cdot \left(\frac{\vec{q}}{m_N}\times
        \vec{v}^\perp\right)\\
    \mathcal{O} _6 &= \left(\vec{S}_\chi \cdot \frac{\vec{q}}{m_N} \right)
        \left(\vec{S}_N \cdot \frac{\vec{q}}{m_N} \right) \\
    \mathcal{O} _7 &= \vec{S}_N\cdot \vec{v}^\perp \\
    \mathcal{O} _8 &= \vec{S}_\chi\cdot \vec{v}^\perp \\
    \mathcal{O} _9 &= i\vec{S}_\chi \cdot \left(\vec{S}_N \times
        \frac{\vec{q}}{m_N}\right)\\
    \mathcal{O} _{10} &= i\vec{S}_N \cdot \frac{\vec{q}}{m_N}\\
    \mathcal{O} _{11} &= i\vec{S}_\chi \cdot \frac{\vec{q}}{m_N}\\
    \mathcal{O} _{12} &= \vec{S}_\chi \cdot \left( \vec{S}_N\times
        \vec{v}^\perp\right)\\
    \mathcal{O} _{13} &= i\left( \vec{S}_\chi \cdot \vec{v}^\perp \right)
        \left(\vec{S}_N\cdot \frac{\vec{q}}{m_N}\right )\\
    \mathcal{O} _{14} &= i\left( \vec{S}_\chi \cdot \frac{\vec{q}}{m_N} \right)
        \left(\vec{S}_N\cdot \vec{v}^\perp \right )\\
    \mathcal{O} _{15} &= -\left(\vec{S}_\chi \cdot \frac{\vec{q}}{m_N} \right )
        \left( \left( \vec{S}_N\times \vec{v}^\perp\right)\cdot
        \frac{\vec{q}}{m_N} \right)
\end{align}

Operator 2 is generally discarded because it is not a leading order
non-relativistic reduction of a manifestly relativistic operator
[@Anand_2014].  Operators 1 and 4 correspond to the naive density- and
spin-coupling, respectively.  

## Nuclear response functions

The EFT physics has been grouped into eight WIMP response functions
$R_i^{x,x'}$, and eight nuclear response functions $W_i^{x,x'}$.
The first six nuclear response functions have
the following form:
\begin{equation}
    W_{X}^{x,x'} = \sum_{J}\bra{\Psi} X^{x}_J \ket{\Psi}\bra{\Psi} X^{x'}_J \ket{\Psi},
\end{equation}
with $X$ selecting one of the six electroweak operators,
\begin{equation}
    X_J=M_J, \Delta_J, \Sigma_J', \Sigma_J'', \tilde{\Phi}_J', \Phi_J'',
\end{equation}
and $\Psi$ being the nuclear wave function for the ground state of the target
nucleus.  The sum over operators spins $J$ is restricted to even or odd values
of $J$, depending on restrictions from conservation of parity and charge
conjugation parity (CP) symmetry.

Two additional response functions add interference-terms:
\begin{equation}
W_{M\Phi''}^{x,x'} =
\sum_{J}\bra{\Psi} M_{J}^{x} \ket{\Psi}\bra{\Psi} \Phi_{J}^{''x'} \ket{\Psi},
\end{equation}
\begin{equation}
W_{\Delta\Sigma'}^{x,x'} = 
\sum_{J}\bra{\Psi} \Sigma_{J}^{'x} \ket{\Psi}\bra{\Psi} \Delta_{J}^{x'} \ket{\Psi}.
\end{equation}
The indices $i$ in equation (\ref{eq:T2}) correspond to these operators as:
$i\to X$ for $i=1,..,6$, and $i=7 \to M\Phi''$, $i=8\to \Delta\Sigma'$.

DMFortFactor can print the nuclear form factors to a file over a range of either
transfer momenta or recoil energy.

## Nuclear (electroweak) operators
There are six parity-and-CP-conserving nuclear operators, $M_J, \Delta_J,
\Sigma_J', \Sigma_J'', \tilde{\Phi}_J', \Phi_J''$, describing the electro-weak
coupling of the WIMPs to the nucleon degrees of freedom.  These are constructed
from Bessel spherical and vector harmonics[@DONNELLY1979103]:
\begin{equation}
    M_{JM}(q\vec{x})\equiv j_J(qx)Y_{JM}(\Omega_x)
\end{equation}
\begin{equation}
    \vec{M}_{JML}(q\vec{x}) \equiv j_L(qx) \vec{Y}_{JLM}(\Omega_x),
\end{equation}
where, using unit vectors $\vec{e}_{\lambda = -1, 0, +1}$,
\begin{equation}
    Y_{JLM}(\Omega_x) = \sum_{m\lambda} \bra{Lm1\lambda}\ket{(L1)JM_J} Y_{Lm}(\Omega_x)\vec{e}_\lambda.
\end{equation}
The six multipole operators are defined as:
\begin{equation}
\begin{split}
M_{JM}\ \ &\\
\Delta_{JM} \equiv& \vec{M}_{JJM}\cdot \frac{1}{q}\vec{\nabla}\\
\Sigma'_{JM} \equiv& -i \left \{\frac{1}{q}\vec{\nabla}\times \vec{M}_{JJM}  \right\}\cdot \vec{\sigma}\\
\Sigma''_{JM} \equiv& \left \{ \frac{1}{q}\vec{\nabla}M_{JM} \right \}\cdot \vec{\sigma}\\
\tilde{\Phi}'_{JM} \equiv& \left( \frac{1}{q} \vec{\nabla} \times \vec{M}_{JJM}\right)\cdot \left(\vec{\sigma}\times \frac{1}{q}\vec{\nabla} \right) + \frac{1}{2}\vec{M}_{JJM}\cdot \vec{\sigma}\\
\Phi''_{JM}\equiv& i\left(\frac{1}{q}\vec{\nabla}M_{JM} \right)\cdot \left(\vec{\sigma}\times \frac{1}{q}\vec{\nabla} \right)
\end{split}
\end{equation}

The matrix elements of these operators can be calculated for standard wave
functions from second-quantized shell model calculations:
\begin{equation}
    \bra{\Psi_f} X_J \ket{\Psi_i} = \Tr(X_J \rho^{f,i}_J )
\end{equation}
\begin{equation}
 = \sum_{a,b} \bra{a} |X_J| \ket{b} \rho^{fi}_J(ab),
\end{equation}
where single-particle orbital labels $a$ imply shell model quantum number $n_a,
l_a, j_a$, and the double-bar $||$ indicates reduced matrix
elements[@edmonds1996angular]. For elastic collisions, only the ground
state is involved, i.e. $\Psi_f=\Psi_i=\Psi_{g.s.}$.

We assume a harmonic oscillator single-particle basis, with the important
convention that the radial nodal quantum number $n_a$ starts at 0, that is, we
label the orbitals as $0s, 0p, 1s0d$, etc..,
and _not_ starting with $1s, 1p,$ etc.
Then, the one-body matrix elements for operators $\bra{a} |X^{(f)}_J| \ket{b}$,
built from spherical Bessel functions and vector spherical harmonics,  have
closed-form expressions in terms of confluent hypergeometric
functions[@DONNELLY1979103].

The nuclear structure input is in the form of one-body density matrices between
many-body eigenstates,
\begin{equation}
\rho^{fi}_J(ab) = \frac{1}{\sqrt{2J+1} }\langle \Psi_f || [ \hat{c}^\dagger_a \otimes \tilde{c}_b ]_J
|| \Psi_i \rangle, \label{eqn:denmat}
\end{equation}
where $\hat{c}^\dagger_a$ is the fermion creation operator (with good angular
momentum quantum numbers), $\tilde{c}_b$ is the
time-reversed[@edmonds1996angular] fermion destruction operator.  Here the
matrix element is reduced in angular momentum but not isospin, and so are in
proton-neutron format. These density matrices are the product of a many-body
code, in our case BIGSTICK[@BIGSTICK1;@BIGSTICK2], although one could use
one-body density matrices, appropriately formatted, from any many-body code.


## Electroweak matrix elements
To compute the matrix elements of the electroweak operators in a harmonic
oscillator basis, we use the derivations from[@DONNELLY1979103]. Namely,
equations (1a) - (1f) and (3a) - (3d), which express the necessary geometric
matrix elements in terms of matrix elements of the spherical Bessel functions.
Here, we write out the remaining explicit formulas for obtaining matrix elements
of the Bessel functions $j_L(y)$ in a harmonic oscillator basis in terms of the
confluent hypergeometric function:
\begin{equation}
    _1F_1(a,b,z) = \sum_{n=0}^\infty \frac{a^{(n)}z^n}{b^{(n)}n!},
\end{equation}
which makes use of the rising factorial function:
\begin{equation}
    m^{(n)} = \frac{(m+n-1)!}{(m-1)!}.
\end{equation}

The first additional relation is computed in DMFortFactor by the function
`BesselElement`:
\begin{equation}
\bra{n'l'j'} j_L(y) \ket{nlj} = \frac{2^L}{(2L+1)!!} y^{L/2} e^{-y}
    \sqrt{(n'-1)!(n-1)!}
    \sqrt{\Gamma(n'+l'+1/2)\Gamma(n+l+1/2)}\\
    \times
    \sum_{m=0}^{n-1}\sum_{m'=0}^{n'-1}
    \frac{(-1)^{m+m'}}{m!m'!(n-m-1)!(n'-m'-1)!} \\
    \times
    \frac{\Gamma[(l+l'+L+2m+2m'+3)/2]}{\Gamma(l+m+3/2)\Gamma(l'+m'+3/2)}
    \ _1F_1[(L-l'-l-2m'-2m)/2; L+3/2; y].
\end{equation}
The two additional relations are needed. As computed by `BesselElementMinus`:
\begin{equation}
\bra{n'l'j'} j_L(y) (\frac{d}{dy}-\frac{l}{y}) \ket{nlj}
    = \frac{2^(L-1)}{(2L+1)!!} y^{(L-1)/2} e^{-y}
    \sqrt{(n'-1)!(n-1)!}
    \sqrt{\Gamma(n'+l'+1/2)\Gamma(n+l+1/2)} \\
    \times
    \sum_{m=0}^{n-1}\sum_{m'=0}^{n'-1}
    \frac{(-1)^{m+m'}}{m!m'!(n-m-1)!(n'-m'-1)!}
    \frac{\Gamma[(l+l'+L+2m+2m'+2)/2]}{\Gamma(l+m+3/2)\Gamma(l'+m'+3/2)} \\
    \times
    \Big\{ -\frac{1}{2}(l+l'+L+2m+2m'+2)\ _1F_1[(L-l'-l-2m'-2m-1)/2; L+3/2; y]\\
    + 2m\ _1F_1[(L-l'-l-2m'-2m+1)/2; L+3/2; y] \Big\}.
\end{equation}
As computed by `BesselElementPlus`:
\begin{equation}
\bra{n'l'j'} j_L(y) (\frac{d}{dy}+\frac{l}{y}) \ket{nlj}
    = \frac{2^(L-1)}{(2L+1)!!} y^{(L-1)/2} e^{-y}
    \sqrt{(n'-1)!(n-1)!}
    \sqrt{\Gamma(n'+l'+1/2)\Gamma(n+l+1/2)} \\
    \times
    \sum_{m=0}^{n-1}\sum_{m'=0}^{n'-1}
    \frac{(-1)^{m+m'}}{m!m'!(n-m-1)!(n'-m'-1)!}
    \frac{\Gamma[(l+l'+L+2m+2m'+2)/2]}{\Gamma(l+m+3/2)\Gamma(l'+m'+3/2)} \\
    \times
    \Big\{ -\frac{1}{2}(l+l'+L+2m+2m'+2)\ _1F_1[(L-l'-l-2m'-2m-1)/2; L+3/2; y]\\
    + (2l+2m+1)\ _1F_1[(L-l'-l-2m'-2m+1)/2; L+3/2; y] \Big\}.
\end{equation}

All remaining electroweak matrix elements can be computed in terms of these
Bessel elements, combined with vector coupling coefficients, etc., as layed out
in the aforementioned reference.


## Wigner vector coupling functions
We implement a standard set of functions and subroutines for computing the
vector-coupling 3-j, 6-j, and 9-j symbols using the Racah alebraic expressions
[@edmonds1996angular]. We distribute the Fortran library for these functions in
their own [Github project](https://github.com/ogorton/wigner) as well.

One method we use to improve  compute time  is to cache Wigner 3-$j$ and 6-$j$
symbols[@edmonds1996angular] (used to evaluate electro-weak matrix
elements) in memory at the start of run-time. As a side effect, our tests show
that this adds a constant compute time to any given calculation of roughly 0.3
seconds in serial execution and uses roughly 39 MB of memory (for the default
table size). As a point of comparison, the $^{131}$Xe example with all-nonzero
EFT coefficients has a run-time of 30 seconds in parallel execution. If we
disable the table caching, the run-time is roughly 150 seconds, 5 times longer.
The size of the table stored in memory can be controlled via the control file
with the keywords `sj2tablemin` and `sj2tablemax`.

For the 3-j symbol, we use the relation to the Clebsh-Gordon vector-coupling
coefficients:
\begin{equation}
    \begin{pmatrix}
        j_1 & j_2 & J\\
        m_1 & m_1 & M
    \end{pmatrix}
    = (-1)^{j_1-j_2-M}(2J+1)^{-1/2}
    (j_1j_2m_1m_2 | j_1 j_2; J, -M).
\end{equation}
The vector coupling coefficients are computed as:
\begin{equation}
    (j_1j_2  m_1m_2 | j_1 j_2; J, M) = \delta(m_1+m_1,m) (2J+1)^{1/2}\Delta(j_1j_2J)\\
     \times[(j_1+m_1)(j_1-m_1)(j_2+m_2)(j_2-m_2)(J+M)(J-M)]^{1/2}\sum_z (-1)^z \frac{1}{f(z)},
\end{equation}
where
\begin{equation}
    f(z) = z!(j_1+j_2-J-z)!(j_1-m_2-z)!
    (j_2+m_2-z)!(J-j_2+m_1+z)!(J-m_1-m_2+z)!,
\end{equation}
and
\begin{equation}
    \Delta(abc) = \left[\frac{(a+b-c)!(a-b+c)!(-a+b+c)!}{(a+b+c+1)!} \right]^{1/2}.
\end{equation}
The sum over $z$ is over all integers such that the factorials are well-defined
(non-negative-integer arguments).

Similarly, for the 6-j symbols:
\begin{equation}
    \begin{Bmatrix}
        j_1 & j_2 & j_3\\
        m_1 & m_1 & m_3
    \end{Bmatrix}
    = \Delta(j_1j_2j_3)\Delta(j_1m_2m_3)\Delta(m_1j_2m_3)
     \Delta(m_1m_2j_3) \sum_z (-1)^z\frac{(z+1)!}{g(z)},
\end{equation}
with
$g(z) = (\alpha - z)!(\beta-z)!(\gamma-z)!(z-\delta)!(z-\epsilon)!(z-\zeta)!(z-\eta)!$ 
and
\begin{equation}
\begin{split}
    \alpha &= j_1+j_1+m_1+m_2 & \beta  &= j_2+j_3+m_2+m_3\\
    \gamma &= j_3+j_1+m_3+m_1 \\
    \delta &= j_1+j_2+j_3 & \epsilon &= j_1+m_2+m_3 \\
    \zeta &= m_1+j_2+m_3 & \eta &= m_1+m_2+j_3.
\end{split}
\end{equation}

For the 9-j symbol, we use the relation to the 6-j symbol:
\begin{equation}
    \begin{Bmatrix}
        j_1 & j_2 & j_3\\
        j_4 & j_5 & j_6\\
        j_7 & j_8 & j_9
    \end{Bmatrix}
    = \sum_k (-1)^{2k} (2k+1)
      \begin{Bmatrix}
        j_1 & j_4 & j_7\\
        j_8 & j_9 & z
      \end{Bmatrix}
      \begin{Bmatrix}
        j_2 & j_5 & j_8\\
        j_4 & z & j_6
      \end{Bmatrix}
      \begin{Bmatrix}
        j_3 & j_6 & j_9\\
        z & j_1 & j_2
      \end{Bmatrix}.        
\end{equation}
The 6-j symbols used to calculate the 9-j symbol are first taken from any
tabulated values. Otherwise, they are computed as previously described.

## Unique couplings

Previous work has focused on setting limits on a single operator coupling at a
time. But of course, multiple couplings may exist simultaneously, and in fact,
some nuclear response functions are only activated with specific pairs of EFT
coefficients.

To create a minimal list of inputs to validate all possible nonzero
couplings, we need to test each coefficient on its own ($i=1,...,15$), and also
test the following 9 unique combinations: (1,2), (1,3), (2,3), (4, 5), (5,6),
(8,9), (11,12), (11,15), (12,15).

Table of EFT coefficient interactions. Shows which coefficients
    multiply each coefficient in addition to itself.

| Coefficient | Couples to |
| ----------- | ---------- |
| 1           |   2, 3   |
| 2           |   1, 3   |
| 3           |   1, 2   |
| 4           |   5, 6   |
| 5           |   4      |
| 6           |   4      |
| 7           |          |
| 8           |   9      |
| 9           |   8      |
| 10          |          |
| 11          |   12, 15 |
| 12          |   11, 15 |
| 13          |          |
| 14          |          |
| 15          |   11, 12 |
