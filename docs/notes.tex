\documentclass{article}
\usepackage[utf8]{inputenc}
\usepackage[a5paper]{geometry}
\usepackage{physics}
\usepackage{graphicx}
\usepackage{listings}
\usepackage[dvipsnames]{xcolor}
\usepackage{hyperref}
\usepackage{multicol}
\usepackage{longtable}
%New colors defined below
\definecolor{codegreen}{rgb}{0,0.6,0}
\definecolor{codegray}{rgb}{0.5,0.5,0.5}
\definecolor{codepurple}{rgb}{0.58,0,0.82}
\definecolor{backcolour}{rgb}{0.95,0.95,0.92}
%Code listing style named "mystyle"
\lstdefinestyle{mystyle}{
  backgroundcolor=\color{backcolour},   commentstyle=\color{codegreen},
  keywordstyle=\color{magenta},
  numberstyle=\tiny\color{codegray},
  stringstyle=\color{codepurple},
  basicstyle=\ttfamily\footnotesize,
  breakatwhitespace=false,
  breaklines=true,
  captionpos=b,
  keepspaces=true,
  numbers=left,
  numbersep=5pt,
  showspaces=false,
  showstringspaces=false,
  showtabs=false,
  tabsize=2
}
\lstset{style=mystyle}


\title{Effective Field Theory Parameters}
\author{Oliver Gorton}
\date{\today}

\begin{document}
\maketitle

In these notes I define the parameters used to make known the effective field
theory interaction used in dmfortfactor. As a shorthand, "the code" will refer to
the dmfortfactor program.

The code uses the EFT coefficients in explicit proton-neutron couplings, i.e.
the interaction is defined by:
\begin{equation}
    \sum_{x=p,n}\sum_{i=1,15} c^x_i \mathcal{O} _i
\end{equation}
and the 15 momentum dependent operators are:
\begin{align}
    \mathcal{O} _1 &= 1_\chi 1_N\\
    \mathcal{O} _2 &= (v^\perp)^2\\
    \mathcal{O} _3 &= i\vec{S}_N \cdot \left(\frac{\vec{q}}{m_N}\times
        \vec{v}^\perp\right)\\
    \mathcal{O} _4 &= \vec{S}_\chi \cdot \vec{S}_N\\
    \mathcal{O} _5 &= i\vec{S}_\chi \cdot \left(\frac{\vec{q}}{m_N}\times
        \vec{v}^\perp\right)\\
    \mathcal{O} _6 &= \left(\vec{S}_\chi \cdot \frac{\vec{q}}{m_N} \right)
        \left(\vec{S}_N \cdot \frac{\vec{q}}{m_N} \right) \\
    \mathcal{O} _7 &= \vec{S}_N\cdot \vec{v}^\perp \\
    \mathcal{O} _8 &= \vec{S}_\chi\cdot \vec{v}^\perp \\
    \mathcal{O} _9 &= i\vec{S}_\chi \cdot \left(\vec{S}_N \times
        \frac{\vec{q}}{m_N}\right)\\
    \mathcal{O} _{10} &= i\vec{S}_N \cdot \frac{\vec{q}}{m_N}\\
    \mathcal{O} _{11} &= i\vec{S}_\chi \cdot \frac{\vec{q}}{m_N}\\
    \mathcal{O} _{12} &= \vec{S}_\chi \cdot \left( \vec{S}_N\times 
        \vec{v}^\perp\right)\\
    \mathcal{O} _{13} &= i\left( \vec{S}_\chi \cdot \vec{v}^\perp \right) 
        \left(\vec{S}_N\cdot \frac{\vec{q}}{m_N}\right )\\
    \mathcal{O} _{14} &= i\left( \vec{S}_\chi \cdot \frac{\vec{q}}{m_N} \right) 
        \left(\vec{S}_N\cdot \vec{v}^\perp \right )\\
    \mathcal{O} _{15} &= -\left(\vec{S}_\chi \cdot \frac{\vec{q}}{m_N} \right )
        \left( \left( \vec{S}_N\times \vec{v}^\perp\right)\cdot 
        \frac{\vec{q}}{m_N} \right)
\end{align}
Each of these operators as written has one of several different sets of
dimensions. To match, each operator coefficient must therefore have a conjugate
set of dimensions to bring $c_x^i \mathcal{O}_i$ to the proper dimensionality. 
The code reads the 15 coefficients $c_x^i$ as dimensionless parameters which
here we will label $a_x^i$. The appropriate scaling factors are then added by a
combination of the WIMP-mass $m_\chi$, the target-nucleus-mass $m_N$, and a
standard-model weak interaction mass scale $m_v=246.2$ GeV.

The 5 parameters with dimension $1$ are (1, 4, 7, 8, 12):
\begin{align}
    c^x_{1, 4, 7, 8, 12} &= \frac{4 m_N m_\chi}{m_v^2} a^x_{1, 4, 7, 8, 12}\\
\end{align}

The 2 parameters with dimension $1/GeV$ are (6, 15):
\begin{align}
    c^x_{6,15} &= \frac{4 m_\chi}{m_v^2} a^x_{6, 15}\\
\end{align}

The 7 parameters with dimension $1/GeV^2$ are (3, 5, 9, 10, 11, 13, 14):
\begin{align}
    c^x_{3, 5, 9, 10, 11, 13, 14} &= \frac{4 m_\chi}{m_Nm_v^2} a^x_{3, 5, 9, 10,
    11, 13, 14}\\
\end{align}

\end{document}
